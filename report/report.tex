% NOTES ON ASSIGNMENT
% TODO 19-23 pages of body text (excluding title page, abstract, table of contents and bibliography)
% TODO appendix, if existing, has to be submitted separately
% LINK project goals https://student.itu.dk/study-administration/project-work/project-goals
% LINK workload and page size https://student.itu.dk/study-administration/project-work/workload-and-project-size

% Preamble
\documentclass[11pt]{article}

\title{Assessing biases in AI generated human settlement data from high-resolution satellite imagery}
\author{Krzysztof Abram, Adam Konstantinovi\'{c}, Arian \v{S}ajina\thanks{Master students of Computer science, ITU Copenhagen} \\
        Supervisor: Vedran Sekara\thanks{NERDS group, assistant professor, ITU Copenhagen}}
\date{\today}

% Packages
\usepackage{amsmath}
\usepackage{lipsum} % for dummy text
\usepackage{graphicx} % for including images
\graphicspath{ {images/} }
\usepackage[backend=biber, style=numeric, date=year, sorting=none, natbib=true, sortcites=true]{biblatex}
\usepackage{hyperref}
\addbibresource{references.bib}

% add shortcut command for github url
\newcommand{\giturl}{\url{https://github.com/ChristopherAbram/ResearchProjectAutumn2020}}

% Document
\begin{document}
    \pagenumbering{gobble} % remove page numbering from first page
    \maketitle
    \vfill
    \includegraphics[width=\textwidth]{itu_logo.jpg}
    \newpage
    % comment out to remove table of contents
%    \tableofcontents
%    \newpage
    \pagenumbering{arabic}

    \section*{Abstract}
        Write abstract last.

    %this is a reference \cite[see][figure 3]{FB-paper}
    %this is a reference \cite{HRSL-dataset}
    %this is a reference \cite{grid3-dataset}

    \section{Motivation}
    %Here we want to identify, define and delimit a problem within information technology.

    Accurately identifying and mapping inhabited areas, and deriving human settlement maps therefrom, is critical
    for humanitarian operations. For example, natural catastrophe risk analysis and efficient planning of vaccination
    campaigns both depend heavily on availability of high-resolution human settlement maps. As the manual process
    of constructing these maps is very costly, and in particular unfeasible for large remote areas that are sparsely
    populated, remote sensing with satellite imagery offers a feasible way to approximate the maps.

    Recent advances in artificial intelligence and price drops in the cost of high-resolution satellite imagery
    have opened possibilities for deriving detailed populations maps from satellite imagery using deep learning frameworks.
    However, while current AI and machine learning algorithms are known to produce satisfactory results in
    classifying well known patterns of settlement areas, it is still a huge challenge to accurately identify
    human settlement patterns in rural and remote areas, and for developing countries where the most vulnerable
    populations live.

    One possible reason for this phenomenon might be the distribution of the data that is used to train the model
    in question. For instance, one would expect the performance of a model trained exclusively on images of buildings
    that are of concrete-brick make, to decrease when evaluated also on tin-roofed buildings.
    Another reason might be the presence of both natural and human-made artifacts such as rock formations or boats,
    which introduce a deterministic source of misclassification.

    The former is of particular significance for developing countries; for instance one study estimates that
    on average in Sub-Saharan Africa (excluding South Africa), in rural areas, only around 45\% of houses were made
    of finished materials\footnote{for example, finished floor materials include parquet, vinyl, ceramic tiles, cement
    and carpet, whereas natural or rudimentary floor materials include earth, sand, dung, wood planks, palm and bamboo}
    \cite[see][Table 1 and Table S1 in supplementary]{Tusting2019}.

    Since the make of a building is strongly correlated to the development level of an area, this study attempts to
    asses if one such human settlement map, constructed by deep learning from satellite imagery, contains biases
    related to socio-economic factors. A conclusive result would better inform any party that takes action based off of
    that particular dataset and possibly quantify the uncertainty with which one should read the different areas of
    the map.


    \section{Introduction}
    Here we want to identify and analyse relevant means for solving the problem, such as academic theories, methods, literature, tools and other sources, as well as existing solutions to the problem.

    \lipsum[1-1]

    \section{Methods}
    Here we want to combine the selected means, develop them further if necessary, and apply them in a concerted effort towards the solution of the problem.

    \lipsum[1-1]

    \section{Results}
    Here we want to evaluate the achieved solution.

    \lipsum[1-1]

    \section{Discussion}
    Here we want to report in a coherent and stringent way the problem, the background research, the work towards the solution, the achieved solution, the evaluation, and other relevant material, while adhering to the academic standards.
    Also, we want to reflect upon the problem, the chosen approach, the achieved solution, and other findings.
    
    \lipsum[1-1]
    
    \section{Acknowledgements}
    Here we want to thank Vedran and Do.

    % print preferences
    \printbibheading
    \printbibliography[type=article,heading=subbibliography,title={Articles}]
    \printbibliography[type=misc,heading=subbibliography,title={Datasets}]

\end{document}